\documentclass[12pt, a4paper]{report}

\usepackage{times}
\usepackage[top=2.5cm, bottom=2.5cm, left=3.5cm, right=2.5cm]{geometry}
\usepackage{subfigure}
\usepackage{pstricks}
\usepackage{pst-node}
\usepackage{graphicx}
\usepackage{caption}
\usepackage{authoraftertitle}
\usepackage[utf8]{inputenc}
\usepackage[T1]{fontenc}
\usepackage{amsmath}
\usepackage{amsfonts}
\usepackage{amssymb}
\usepackage[lined,ruled,vlined,linesnumbered]{algorithm2e}
\usepackage{xspace,epsfig,url}
\usepackage{pst-plot,pstricks-add}
\usepackage{etoolbox}
\usepackage{setspace}
\usepackage{float}
\usepackage{multirow}
\usepackage[section]{placeins}
\usepackage{setspace}
\usepackage{subfiles}
\usepackage{titlesec}
\usepackage{sectsty}


\newcommand{\kkcomm}[1]{{\color{red}{\bf kk: #1}}}

\chapternumberfont{\centering\normalsize\MakeUppercase} 
\chaptertitlefont{\centering\normalsize\MakeUppercase}
\sectionfont{\centering}

\makeatletter
\AtBeginDocument{%
  \expandafter\renewcommand\expandafter\subsection\expandafter{%
    \expandafter\@fb@secFB\subsection
  }%
}

\AtBeginEnvironment{algorithm}{\setstretch{1}}

\SetKwFor{ForEachP}{foreach}{in parallel do}{end}

\newtheorem{theorem}{Theorem}
\newtheorem{definition}[theorem]{Definition}


\newtheorem{proposition}{Proposition}[section]
%\newtheorem{cor}[thm]{Corollary}

%\newcommand{\comment}[2]{{\color{red}{\bf (#1: #2)}}}
\newcommand{\greedyAlgo}{\textsc{Greedy}}
\newcommand{\NPHARD}{{\tt NP-hard}}
\newcommand{\coNPHARD}{{\tt coNP-hard}}

\linespread{1.5}

\newcommand{\ics}[2]{\langle #1, #2 \rangle}
%\newcommand{\ics}[2]{\frac{#1}{#2}}

\title{ALGORITHMIC OPTIMIZATION AND PARALLELIZATION OF EPPSTEIN'S SYNCHRONIZING HEURISTIC}

\author{SERTAÇ KARAHODA}

\date{July, 2108}

\begin{document}

\pagenumbering{gobble}

\begin{center}

\vspace*{6\baselineskip}

\MyTitle

\vspace*{6\baselineskip}

by

\MyAuthor

\vspace*{6\baselineskip}

Submitted to the Graduate School of Sabancı University \\
in partial fulfillment of the requirements for the degree of \\
Master of Science

\vspace*{1\baselineskip}

Sabancı University

July, 2018
\end{center}


\clearpage
$ $
\thispagestyle{empty}
\clearpage
$ $
\vspace{5cm}
\begin{center}
\copyright \hspace{0.1cm} \MyAuthor\ 2018

All Rights Reserved
\thispagestyle{empty}
\end{center}
\clearpage



\pagenumbering{roman}
\setcounter{page}{4}


\begin{center}
{\bf ABSTRACT}
\end{center}

\begin{center}
\large
\MyTitle
\end{center}

\begin{center}
\MyAuthor

Computer Science and Engineering, Master's Thesis, 2018

Thesis Supervisor: Hüsnü Yenigün\\
Thesis Co--Supervisor: Kamer Kaya
\end{center}

\begin{center}
Keywords: Finite state automata, Synchronizing words, Synchronizing heuristics, CPU parallelization, GPU parallelization
\end{center}

\begin{quote}

\begin{spacing}{1}
Testing is the most expensive and time consuming phase in the development of complex systems. Model--based testing is an approach that can be used to automate the generation of high quality test suites, which is the most challenging part of testing. Formal models, such as finite state machines or automata, have been used as specifications from which the test suites can be automatically generated. The tests are applied after the system is synchronized to a particular state, which can be accomplished by using a synchronizing word.  Computing a shortest synchronizing word is of interest for practical purposes, e.g. for a shorter testing time. However, computing a shortest synchronizing word is an NP--hard problem. Therefore, heuristics are used to compute short synchronizing words. \textsc{Greedy} is one of the fastest synchronizing heuristics currently known. In this thesis, we present approaches to accelerate \textsc{Greedy} algorithm. Firstly, we focus on parallelization of \textsc{Greedy}. Second, we propose a lazy execution of the preprocessing phase of the algorithm, by postponing the preparation of the required information until it is to be used in the reset word generation phase. We suggest other algorithmic enhancements as well for the implementation of the heuristics. Our experimental results show that depending on the automata size, \textsc{Greedy} can be made $500\times$ faster. The suggested improvements become more effective as the size of the automaton increases.
\end{spacing}
\end{quote}

\clearpage

\begin{center}
{\bf ÖZET}
\end{center}

\begin{center}
\large
EPPSTEIN'IN SIFIRLAMA SEZGİSELİNİN ALGORİTMİK ENİYİLEMESİ VE PARALELLEŞTİRİLMESİ
\end{center}

\begin{center}
\MyAuthor

Bilgisayar Bilimi ve Mühendisliği, Yüksek Lisans Tezi, 2018

Tez Danışmanı: Hüsnü Yenigün

Tez Eşdanışmanı: Kamer Kaya
\end{center}

\begin{center}
Anahtar Kelimeler: Sonlu durum otomatları, Sıfırlama kelimeleri, Sıfırlama Sezgiselleri, AİÜ paralelleştirilmesi, GİÜ paralelleştirilmesi
\end{center}

\begin{quote}
\begin{spacing}{1}
Karmaşık sistemlerin geliştirilmesinde, test etme en pahalı  ve en çok zaman alan evredir. Model tabanlı testler yüksek kaliteli deney kurgusunu otomatik üretmede kullanılan yaklaşımlardan birisidir. Deney kurgusunu otomatik\linebreak üretme test etmenin en zorlu parçalarından biridir. Sonlu durum makineleri ya da özdevinimler gibi biçimsel modeller, otomatik deney grubunu üretmek için kullanılmaktadır. Sistem  belirli bir duruma senkronize edildikten sonra testler uygulanır ve bu belirli duruma gelebilmek için sıfırlama kelimeleri kullanılmaktadır. Daha kısa deney süreleri için en kısa sıfırlama kelimesini hesaplamak önemlidir, ancak en kısa sıfırlama kelimesini hesaplamak NP--hard bir problemdir. Bu nedenle kısa sıfırlama kelimelerini hesaplamak için sezgisel yöntemler kullanılmaktadır. \textsc{Greedy} algoritması bu alanda bilinen en hızlı sezgisel algoritmadır. Bu tezde, \textsc{Greedy} algoritmasını hızlandıran yaklaşımlar sunulmaktadır. İlk olarak \textsc{Greedy} algoritmasının paralelleştirilmesine odaklanılmaktadır. İkinci olarak ise tembel bir yaklaşım önererek sıfırlama kelimesinin üretilmesi için gerekli bilgilerin hazırlanma süreci ertelenmektedir. Aynı zamanda, \textsc{Greedy} algoritması için benzer algoritmik iyileştirilmeler önerilmektedir. Deney sonuçlarımız özdevinim büyüklüğüne bağlı olarak \textsc{Greedy} algoritmasının 500 kat daha hızlı hale getirilebileceğini göstermektedir. önerilen geliştirmeler özdevinim büyüklüğü arttıkça daha etkili hale gelmektedir.
\end{spacing}
\end{quote}
\clearpage
$ $
\vspace{2cm}
\begin{center}
\textbf{ACKNOWLEDGMENTS}
\end{center}

I would like to state my gratitude to my supervisors, Hüsnü Yenigün and Kamer Kaya for everything they have done for me, especially for their invaluable guidance, limitless support and understanding.

The financial support of Sabanci University is gratefully acknowledged.

I would like to thank TUBITAK 114E569 project for the financial support provided.


\clearpage

\begin{spacing}{1}
\tableofcontents
\end{spacing}

\clearpage 

\begin{spacing}{1}
\listoffigures
\end{spacing}

\clearpage

\begin{spacing}{1}
\listoftables
\end{spacing}

\clearpage

\begin{spacing}{1}
\listofalgorithms
\end{spacing}


\clearpage
\pagenumbering{arabic}

\subfile{sections/intro}

\clearpage
\subfile{sections/preliminaries}

\clearpage
\subfile{sections/greedy}


\clearpage
\subfile{sections/parallel}

\clearpage
\subfile{sections/speedup}

\clearpage
\subfile{sections/results}

\clearpage
\subfile{sections/conclusion}

\clearpage

\bibliographystyle{siam}
\begin{spacing}{1}
\bibliography{thesis}
\end{spacing}

\end{document}


