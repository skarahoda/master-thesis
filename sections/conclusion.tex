\chapter{CONCLUSION AND FUTURE WORK}
\label{sec:conclusion}

\textsc{Greedy} is widely used as a performance baseline for the new heuristics in the literature. None of the newer heuristics could actually match the time performance of \textsc{Greedy} and \textsc{Cycle} so far. With the new improved approaches suggested in this thesis, \textsc{Greedy} and \textsc{Cycle} have now become more competitive than they already are.

We investigated the efficient implementation and use of modern multicore CPUs and GPUs to scale the performance of synchronizing sequence generation heuristics. We mainly focused on the PMF generation phase (which is employed by almost all the heuristics in the literature), since it is the most time consuming part of \textsc{Greedy}, which is probably the same with \textsc{Cycle}. Even with no parallelization, our algorithmic improvements yield 33x speedup on \textsc{Greedy} for automata with 8000 states and 128 inputs. Furthermore, around 494x speedup has been obtained with GPU parallelization for the same automata class.

First phase of \greedyAlgo\ is common for the well known synchronizing heuristics, such as \textsc{SynchroP}, \textsc{SynchroPL}, and \textsc{FastSynchro}. Since they are more expensive, their first phase does not dominate as much as it does for \textsc{Greedy} and \textsc{Cycle}. However, the parallelization techniques can be applied to other heuristics. One can also parallelize second phase of the slower heuristics to make them more competitive. Furthermore, we proposed techniques to speedup \textsc{Greedy} without additional parallelization. With these optimizations, we obtained order(s) of magnitude faster heuristics. The techniques suggested in this thesis become more effective as the size of the automata increases. Due to the increased speeds of \textsc{Greedy}, the heuristic will now scale more.

\pagebreak

Note that other synchronizing heuristics existing in the literature such as \textsc{SynchroP}, \textsc{SynchroPL}, and \textsc{FastSynchro} cannot directly benefit from the lazy computation and further techniques. These heuristics require shortest merging sequence information for many pairs of states, if not for all. Due to this property, a large part of, or the entire, BFS forest will have to be constructed. Therefore, the underlying intuition for  \textsc{SynchroP}, \textsc{SynchroPL}, and \textsc{FastSynchro} will need to be changed in order for them to take advantage of the algorithmic improvements that the thesis proposed. Modifying the underlying intuition for \textsc{SynchroP}, \textsc{SynchroPL}, and \textsc{FastSynchro} so that they can also benefit from the techniques suggested in this thesis can be considered as a future work.
